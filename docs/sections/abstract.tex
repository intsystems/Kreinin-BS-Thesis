\begin{center}
    \Large{\textbf{Аннотация}}
\end{center}

Исследуется задача минимизации целевой функции потерь. Рассматривается проблема оптимизации целевой функции потерь градиентными методами первого порядка. Исследуется сходимость методов градиентной оптимизации с предобуславливанием, использующих регуляризацию с затуханием весов. 
Специально рассматриваются популярные методы оптимизации из данного класса методов такие как AdamW и OASIS.
Исследуются различные альтернативы этим методам с целью изучения их скорости сходимости и точности, показываемой моделью.
Предлагается новый способ добавления регуляризации в метод оптимизации Adam. 
Доказывается теорема о скорости сходимости данных методов при различных допущениях на функцию потерь и показывается сходимость к исходной функции потерь.
Проводятся вычислителньые эксперименты с различными эталонными наборами данных, моделями и проводится анализ гиперпараметров, чтобы сравнить их на реальных задачах.