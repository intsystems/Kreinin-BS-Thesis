\subsection{Решение методов с предобуславливанием и затуханием весов}

В предыдущем подразделе мы доказали сходимость методов с предобуславливанием, однако выше мы указали, что методы с затуханием весов сходятся к исходному решению задачи оптимизации \eqref{F_big} $w^*$, а к исходному решению $\widetilde{w}^*$ задачи \eqref{eq:general}, это достигается за счет следующего. Мы получили новую целевую функцию потерь \label{F_tilde_problem}, в которой величина регуляризации динамически изменяется от шага к шагу, на основании матрицы предобуславливания, учитывая, что матрица составляется на основе стохастических градиентов, полученных в ходе обновления весов модели, получается, что регуляризация получается тем больше, чем больше градиент по данному весу модели, и наоборот тем меньше, чем меньше градиент по весу модели. То есть регуляризация не штрафует веса модели, где градиент вышел на значения близкие к нулю. То есть мы стараемся выйти делать больший шаг там, где стохастический градиент не приблизился к каким-то околнулевым значениям, то есть пока мы не оказались в окрестности какого-то экстремума. За счёт этого получается более разнообразная траектория обновления весов модели, которая позволяет нам получать лучшую сходимость на практике. Эти рассуждения подтверждаются экспериментами, подробно описанными в разделе \ref{Experiments}.

Оценим разницу между решениями задач \eqref{F_big} и \eqref{F_tilde}. Это ограничение основано на предположениях \eqref{ass:smoothness} и свойствах матрицы $D_t$.

\begin{lemma}\label{lemma:lowerbondF}{(Lower bound)}
Предполагая, что \ref{ass:regstruct}, \ref{ass:precondstruct} и \ref{ass:smoothness} выполняются, также предполагая, что задачи \eqref{F_tilde} и \eqref{F_big} имеют соответствующие решения $\widetilde{w}^*$ и $w^*$, тогда разница между решениями может быть ограничена снизу:
    \begin{equation*}
        \|\widetilde{w}^* - w^* \| L_F \geq \| \nabla r (\widetilde{w}^*) (I - D_t)\|.
    \end{equation*}
\end{lemma}

Следовательно, можно заметить, что использование регуляризации весов не в прямом подсчете градиента, которое влечет и учет их в матрице предобуславливания приводит к сходимости к решению исходной задачи, в то время, как прямое использование функции регуляризации для подсчета стохастического градиента приводит нас к альтернативному решению. Расхождение между этими решениями зависит от нормы разности между $D_t$ и матрицей тождества ($||D_t - I||$).
В результате анализ распределения элементов $D := \lim\limits_{t \to \infty} D_t$ может дать представление о сходимости метода с затуханием весов.

