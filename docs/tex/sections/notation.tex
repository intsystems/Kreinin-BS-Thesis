\section{Основная часть}

\subsection{Обозначения}
\begin{itemize}
    \item Мы использкем $x^i$, где $x$ это вектор и $i \in \overline{1, d}$ обозначает $i$-ю компоненту $d$-мерного вектора $x$.
    \item Для любых $x, y \in \mathbb{R}^d$ скалярное произведение обозначается, как $\langle x, y \rangle := \sum_{i=1}^d x^i y^i$.
    \item $L_f$ -- константа липшица функции $f$, то есть $\forall x, y \in \mathbb{R}^d \rightarrow $ 	$f(x) \leq f(y) + \langle \nabla f(y), x-y \rangle + \frac{L_f}{2} \|x - y\|^2$
    \item $|| x || := \sqrt{\langle x, x \rangle},$ где $x \in \mathbb{R}^d$ это $l_2$ норма вектора $x$.
    \item $|| x ||_A^2 := x^TAx$, где $x \in \mathbb{R}^d, A \in \mathbb{R}^{d \times d}$
    \item Для матрицы $A \in \mathbb{R}^{d \times d}$, $A^{-1}$ -- обратная матрица.
    \item Мы используем $A \preccurlyeq B$ для двух матриц $A, B \in \mathbb{R}^{d \times d}$, чтобы обозначить что $x^TAx \le x^TBx$ для любых $x \in \mathbb{R}^d$.
    \item $\textrm{diag} \left\{ \beta_1 \ldots, \beta_d \right\}$ -- диагональная матрица, состоящая из элементов: $\beta_1, ..., \beta_d \in \mathbb{R}.$
    
\end{itemize}
