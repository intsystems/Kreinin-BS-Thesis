\appendix
\section{Приложение}
\subsection{Доказательство лемм}
\label{appendix:lemmas}

\begin{proof} (Лемма \ref{lemma:Dt})
\begin{enumerate}
    \item Благодаря выражению \eqref{eq:alpha} $\hat{D}_t$ ограничена снизу. Из уравнений обновления матрицы предобуславливания \eqref{eq:linear} и \eqref{eq:alpha} можно сделать вывод, что условие \ref{ass:precondstruct} выполняется из итерационного обновления матрицы, а также абсолютное значение диагональных элементов матрицы $D_t$ ограниченно сверху после каждого обновления, по причине того, что матрица $H_t$ ограниченна сверху.

    \item Можем ограничить норму разности сверху, используя \eqref{eq:squares} и тот факт, что все матрицы диагональные
    \begin{equation*}
    \begin{aligned}
    || \hat{D}_{t+1} - \hat{D}_t||_\infty \le || D_{t+1} - D_t||_\infty &=  || ((D_{t+1})^2 - (D_t)^2)(D_{t+1} + D_t)^{-1} ||_\infty \\
    &\le (1 - \beta)|| ((H_t)^2 - (D_t)^2)(D_{t+1} + D_t)^{-1} ||_\infty \\
    &\le (1 - \beta)\frac{\Gamma^2}{2\alpha}.
    \end{aligned}
    \end{equation*}

    Ограничили каждый фактор по отдельности: $||(D_t)^2 - (H_t)^2||_\infty$ ограничен $\Gamma^2$, так как каждый из диагональных элементов $(D_t)^2$ и $(H_t)^2$ больше 0 и меньше $\Gamma^2$.
    Второй коэффициент ограничен, поскольку и $D_{t+1}$, и $D_t$ больше или равны $\alpha$, как доказано в первом утверждении этой леммы, следовательно, $$(D_{t+1} + D_t)^{-1} \preccurlyeq \frac{1}{2\alpha}.$$

    \item Можем ограничить норму разности сверху, используя \eqref{eq:linear}, аналогично доказательству второго утверждения этой леммы
    \begin{equation*}
    || \hat{D}_{t+1} - \hat{D}_t||_\infty \le || D_{t+1} - D_t||_\infty \le (1 - \beta)|| H_t - D_t ||_\infty \le 2\Gamma (1 - \beta).
    \end{equation*}
    Используем первое утверждение этой леммы об ограниченности диагональных элементов.
\end{enumerate}
\end{proof}

\begin{proof} (Лемма \ref{lemma:existence})

Используя предположения \ref{ass:regstruct}, \ref{ass:precondstruct}, мы можем записать градиент $\widetilde{r}$
\begin{equation*}
\nabla \widetilde{r} = \nabla \left( \sum_{i=1}^d D_t^i r_i(w_i) \right) = D_t \begin{pmatrix}
  r_1'(w_1) \\
  \vdots  \\
  r_d'(w_d)
\end{pmatrix} = D_t \nabla r.
\end{equation*}
Доказали необходимое на утверждение.
\end{proof}

\begin{proof} (Лемма \ref{lemma:tildesmoothness})
Мы можем написать определение гладкости, используя лемму \ref{lemma:existence}, а затем применить \ref{ass:smoothness} и \ref{ass:preconditioned}.
\begin{equation*}
\begin{split}
    || \nabla \widetilde{r}(x) - \nabla \widetilde{r}(y) || &=
    \Bigg{\|} \nabla \left( \sum_{i=1}^d D_t^i r_i(x_i) \right) - \nabla \left( \sum_{i=1}^d D_t^i r_i(y_i) \right) \Bigg{\|} \\
    & = ||D_t \left( \nabla r(x) - \nabla r(y) \right)|| \le  ||D_t|| L_r \le \Gamma L_r
\end{split}
\end{equation*}
Доказали необходимое нам утверждение.
\end{proof}

\begin{proof} (Proof of lemma \ref{lemma:lowerbondF})
    Напишем определения решений $w^*$, $\widetilde{w}^*$:
        \begin{equation*}
        \begin{cases}
            \nabla f (\widetilde{w}^*) + D_t \nabla r(\widetilde{w}^*) = 0\\
            \nabla f (w^*) + \nabla r(w^*) = 0\\
        \end{cases},
        \end{equation*}
        Тогда мы можем получить нижнюю границу из определения $L_F$-контрастности функции $F$.
        \begin{equation*}
        \begin{split}
        \|\widetilde{w}^* - w^* \| L_F &\geq \| \nabla f (\widetilde{w}^*) + \nabla r(\widetilde{w}^*) - \nabla f (w^*) - \nabla r (w^*) \| \\
        &= \| - D_t \nabla r (\widetilde{w}^*) + \nabla r(\widetilde{w}^*) \|  = \| \nabla r (\widetilde{w}^*) (I - D_t)\|.
        \end{split}
        \end{equation*}
    Доказали утверждение.
\end{proof}

\subsection{Доказательства теорем}
\label{appendix:theorems}

\begin{proof} (Теорема \ref{theor:1})
\label{proof:theorem1}

Используем предположение \eqref{ass:smoothness} для шагов $t$ и $t+1$:
\begin{equation} \label{eq1}
    f(w_{t+1}) \leq f(w_t) + \langle \nabla f(w_t), w_{t+1} - w_t \rangle + \frac{L_f}{2}||w_{t+1} - w_t ||^2,
\end{equation}
По определению нашего алгоритма мы имеем:
\begin{equation*}
w_{t+1} - w_t = -\eta D_t^{-1} \nabla f(w_t) - \eta \nabla r(w_t).
\end{equation*}
Из предыдущего выражения выбираем градиент функции
\begin{equation*}
\nabla f(w_t) = \frac{1}{\eta} D_t(w_t - w_{t+1}) - D_t \nabla r(w_t),
\end{equation*}
заменим $\nabla f(w_t)$ в \ref{eq1} и по предположению \ref{ass:preconditioned},, $I \preccurlyeq \frac{D_t}{\alpha}$
\begin{equation*}
\begin{split}
    f(w_{t+1}) &\leq f(w_t) + \langle \frac{1}{\eta}D_t(w_t - w_{t+1}) - D_t\nabla r(w_t), w_{t+1} - w_t \rangle + \frac{L_f}{2 \alpha} ||w_{t+1} - w_t||_{D_t}^2 \\
    &= f(w_t) + \left(\frac{L_f}{2 \alpha} - \frac{1}{\eta} \right) ||w_{t+1} - w_t||_{D_t}^2 - \langle D_t \nabla r(w_t), w_{t+1} - w_t \rangle,
\end{split}
\end{equation*}
используя обозначение $\widetilde{r}_t : \nabla \widetilde{r}_t = D_t \nabla r(w_t)$,
мы можем переписать шаг, используя переменную и предположение \eqref{ass:smoothness}
\begin{equation*}
    \widetilde{r}(w_{t+1}) \leq \widetilde{r}(w_t) + \langle \nabla \widetilde{r}(w_t), w_{t+1} - w_t \rangle + \frac{L_{\tilde{r}}}{2} ||w_{t+1} - w_t||_2^2.
\end{equation*}
Заменим старую функцию регуляризации на новую
\begin{equation*}
    f(w_{t+1}) \leq f(w_t) + \left( \frac{L_f}{2\alpha} - \frac{1}{\eta} \right) ||w_{t+1} - w_t||_{D_t}^2 + \tilde{r}(w_t) - \tilde{r}(w_{t+1}) + \frac{\Gamma L_{\tilde{r}}}{2}||w_{t+1}-w_t||_{D_t}^2.
\end{equation*}
Теперь давайте определим новую функцию потерь
$\widetilde{F}_t(w) := f(w) + \tilde{r}_t(w)$, ($\tilde{L}=L_f + \Gamma L_{r}$), мы получаем:
\begin{equation*}
    \widetilde{F}_t(w_{t+1}) \leq \widetilde{F}_t(w_t) + \left( \frac{\widetilde{L}}{2\alpha} - \frac{1}{\eta}  \right) ||w_{t+1} - w_t||_{D_t}^2,
\end{equation*}
мы выбираем шаг таким образом, чтобы $ \frac{\tilde{L}}{2\alpha} - \frac{1}{\eta} < 0 \Leftrightarrow \eta < \frac{2 \alpha}{\tilde{L}}$
\begin{equation}
\label{eq:theor1}
    \left(\frac{1}{\eta} - \frac{\tilde{L}}{2\alpha}   \right) ||w_{t+1} - w_t||_{D_t}^2 \leq \tilde{F}_t(w_t) - \tilde{F}_t(w_{t+1}).
\end{equation}
Тогда заметим, что согласно алгоритму
\begin{equation*}
\begin{aligned}
    ||w_{t+1} - w_t||^2_{D_t} &= ||-\eta D_t^{-1} \nabla f(w_t) - \eta \nabla r(w_t)||_{D_t}^2 \\
    &= \eta^2 || D_t^{-1} 
    ( \nabla f(w_t) + \nabla \widetilde{r}_t(w_t) ) ||_{D_t}^2 \\
    &= \eta^2  ( \nabla f(w_t) + \nabla \widetilde{r}_t(w_t))^T D_t^{-1} D_t  D_t^{-1} 
    ( \nabla f(w_t) + \nabla \widetilde{r}_t(w_t) ) \\
    &\ge \frac{\eta^2}{\Gamma} || \nabla f(w_t) + \nabla \widetilde{r}_t(w_t) ||^2 = \frac{\eta^2}{\Gamma} ||\nabla\widetilde{F}_t(w_t)||^2,
\end{aligned}
\end{equation*}
и заменим $||w_{t+1} - w_t||^2_{D_t}$ в \eqref{eq:theor1} предыдущим уравнением и получим
\begin{equation}
\label{eq:theor1-1}
    \left(\frac{1}{\eta} - \frac{\tilde{L}}{2\alpha}   \right) \frac{\eta^2}{\Gamma} || \nabla\widetilde{F}_t(w_t) ||^2 \leq \tilde{F}_t(w_t) - \tilde{F}_t(w_{t+1}).
\end{equation}
Чтобы получить уравнение, мы связываем норму разности $\tilde{F}_{t+1}$ и $\tilde{F}_t$ при равных переменных $w$:
\begin{equation*}
\begin{aligned}
    | \widetilde{F}_{t+1}(w) - \widetilde{F}_{t}(w)| &= |\widetilde{r}_{t+1}(w) - \widetilde{r}_t(w) | = \left|\sum\limits_{i=0}^d (d_{t+1}^i - d^i_t)r_i(w^i) \right| \\
    &\leq \sum\limits_{i=0}^d |d_{t+1}^i - d^i_t| r_i(w^i) \leq ||D_{t+1} - D_t||_\infty |r(w)| \\
    &\leq \Omega ||D_{t+1} - D_t||_\infty,
\end{aligned}
\end{equation*}
где мы используем предположение \ref{ass:regstruct} и предположение \ref{ass:regbound}.

Затем нам нужно оценить $| \widetilde{F}_{t+1}(w) - \widetilde{F}_{t}(w)|$, используя лемму \ref{lemma:Dt}.
Мы связываем последнее уравнение с $\delta$ и уточняем $\delta$ для случаев \eqref{eq:squares} и \eqref{eq:linear}
\begin{equation}
\label{eq:Ft+1-Ft}
| \widetilde{F}_{t+1}(w) - \widetilde{F}_{t}(w)| \le \Omega ||D_{t+1} - D_t||_\infty \le \delta,
\end{equation}
где $\delta = \begin{cases}
    \frac{(1 - \beta)\Gamma^2}{2\alpha}\Omega & \text{для } \eqref{eq:squares} \\
    2(1 - \beta)\Gamma\Omega  &  \text{для } \eqref{eq:linear}
\end{cases}.$

Теперь мы можем оценить следующую разность, используя \eqref{eq:theor1-1} и \eqref{eq:Ft+1-Ft}.
\begin{equation*}
\begin{aligned}
\widetilde{F}_t(w_t) - \widetilde{F}_{t+1}(w_{t+1}) &= \widetilde{F}_t(w_t) - \widetilde{F}_{t}(w_{t+1}) + \widetilde{F}_{t}(w_{t+1}) - \widetilde{F}_{t+1}(w_{t+1}) \\ 
&\ge \left(\frac{1}{\eta} - \frac{\tilde{L}}{2\alpha}   \right) \frac{\eta^2}{\Gamma} || \nabla\widetilde{F}_t(w_t) ||^2 - \delta,
\end{aligned}
\end{equation*}
и перепишем
\begin{equation*}
 \left(\frac{1}{\eta} - \frac{\tilde{L}}{2\alpha}   \right) \frac{\eta^2}{\Gamma} || \nabla\widetilde{F}_t(w_t) ||^2 \le \widetilde{F}_t(w_t) - \widetilde{F}_{t+1}(w_{t+1}) + \delta.
\end{equation*}
Теперь просуммируем все итерации предыдущего выражения
\begin{equation*}
\begin{split}
    \frac{\eta^2  (T+1)}{\Gamma}\left(\frac{1}{\eta} - \frac{\tilde{L}}{2\alpha}   \right)\cdot\min_{t \in \overline{0, T}} ||\nabla\widetilde{F}_t(w_t)||^2 & \leq \frac{\eta^2}{\Gamma}\left(\frac{1}{\eta} - \frac{\tilde{L}}{2\alpha}   \right)\cdot\sum\limits_{t = 0}^T ||\nabla\widetilde{F}_t(w_t)||^2 \\
    & \leq \tilde{F}(w_0) - \tilde{F}^* + \delta \cdot (T+1)
\end{split}
\end{equation*}
Переместив все в правую часть, мы получим следующую оценку
\begin{equation*}
    \min_{t \in \overline{0, T}} ||\nabla f(w_t) + \nabla \tilde{r}(w_t)||^2 \leq \frac{(\tilde{F}(w_0) - \tilde{F}(w_*))\Gamma}{(\frac{1}{\eta} - \frac{\tilde{L}}{2\alpha}) \eta^2 (T+1)} + \frac{\delta \Gamma}{(\frac{1}{\eta} - \frac{\tilde{L}}{2\alpha}) \eta^2} = \varepsilon,
\end{equation*}
\begin{equation*}
    T + 1 \geq \frac{\Delta_0 \Gamma}{(\eta - \frac{\tilde{L}\eta^2}{2\alpha}) \left( \varepsilon -\frac{\delta\Gamma}{\eta - \frac{\tilde{L}\eta^2}{2\alpha}}\right)}.
\end{equation*}
Мы получаем оценку количества шагов, необходимых для достижения заданной точности
\begin{equation*}
      T = \mathcal{O}\left( \frac{\Delta_0 \Gamma}{(\eta - \frac{\tilde{L}\eta^2}{2\alpha}) \left( \varepsilon -\frac{\delta\Gamma}{\eta - \frac{\tilde{L}\eta^2}{2\alpha}}\right)} \right).
\end{equation*}
\end{proof}

\begin{proof} (Теорема \ref{theor:2})
В дальнейшем доказательстве нам понадобятся вспомогательный термин - $\sigma_{t+1}^2$, он понадобится нам для записи рекурсии:
\begin{equation*}
    \sigma_{t+1}^2 = ||D_{t+1}||_2^2 = ||\beta D_{t} + (1-\beta) H_t ||_2^2 =\beta^2 ||D_{t} +\frac{1-\beta}{\beta}H_t||_2^2.
\end{equation*}
Тогда давайте перепишем:
\begin{equation*}
\begin{split}
    \sigma_{t+1}^2 &\leq \beta^2 \dot (1+\frac{1}{a}) \sigma_{t}^2 + \left(\frac{1-\beta}{\beta}\right)^2(1+a)||\nabla f(w_t)||_2^2 \\ 
    & =  \beta^2 (1+\frac{1}{a})\sigma_{t}^2 + \left(\frac{1-\beta}{\beta} \right)^2(1+a) ||\nabla f(w_t) - \nabla f(w_*)||_2^2  \\
    & \leq \beta^2(1+\frac{1}{a})\sigma_{t}^2 + 2\left(\frac{1-\beta}{\beta} \right)^2(1+a)L_f (f(w_t) - f(w_*)).
    \end{split}
\end{equation*}
Выберем $a = \frac{\beta}{1-\beta}$, чтобы получить $\beta^2(1+\frac{1}{a}) = \beta$, учтем это в уравнении $\sigma^2$:
\begin{equation*}
    \sigma_{t+1}^2 \leq \beta \sigma_{t}^2 + 2\frac{1-\beta}{\beta^2}L_f (f(w_t) - f(w_*)).
\end{equation*}
Запишем норму между текущими весами и решением исходной задачи:
\begin{equation*}
    ||w_{t+1}-w_*||_{D_t}^2 = ||w_t - w_*||_{D_t}^2 - 2 \eta \langle \nabla f(w_t) + D_t \nabla r(w_t), w_t - w_* \rangle + \eta^2 ||\nabla f(w_t) +D_t \nabla r(w_t)||_{D_t^{-1}}^2.
    \end{equation*}
С предположением \ref{ass:muconvex} на $f$:
\begin{equation*}
\begin{split}
    ||w_{t+1}-w_*||_{D_t}^2 &\leq ||w_t-w_*||_{D_t}^2 + 2\eta \left(f(w_*) - f(w_t) - \frac{\mu_f}{2} ||w_t - w_*||_2^2 \ \right)\\ 
    & - 2\eta \langle \nabla r(w_t), w_t - w_* \rangle_{D_t} + \eta^2 ||\nabla f(w_t) + D_t \nabla r(w_t)||_{D_t^{-1}}^2.
\end{split}
\end{equation*}
В случае регуляризации $\ell_2$ можем записать третий член:
\begin{equation*}
\begin{split}
    -2\eta \langle D_t \nabla r(w_t), w_t-w_*\rangle &= -2\lambda\eta \langle D_t w_t, w_t-w_*\rangle \\
    &= -2\lambda\eta \langle w_t - w_*, w_t-w_*\rangle_{D_t} - 2\lambda\eta \langle w_*, w_t-w_*\rangle_{D_t}\\
    &= -2\eta\lambda||w_t-w_*||_{D_t}^2 - 2\lambda\eta\langle w_*\sqrt{D_t}, \sqrt{D_t}(w_t-w_*)\rangle \\
    &\leq -2\eta \lambda ||w_t-w_*||_{D_t}^2 + \lambda\eta||w_*\sqrt{D_t}||_2^2 + \lambda\eta||w_t - w_*||_{D_t}^2 \\
    &\leq -\eta\lambda||w_t-w_*||_{D_t}^2 + \frac{\lambda\eta\Omega_0^2}{\alpha}||D_t||_2^2.
    \end{split}
\end{equation*}
Здесь мы использовали, что \ref{ass:preconditioned}, $\alpha I \preccurlyeq D_t \preccurlyeq \Gamma I$ и $||w_*|||_2^2 \leq \Omega_0^2$. С помощью леммы \ref{lemma:existence} и $L$-гладкости \ref{ass:smoothness} функции $f$, $||\nabla f(w_t) - \nabla f(w_*)||_2^2 \leq 2L_f(f(w_t) - f(w_*))$:
\begin{equation*}
\begin{split}
    \eta^2 ||\nabla f(w_t) + D_t \nabla r(w_t)||_{D_t^{-1}}^2 &\leq 2\eta^2 ||\nabla f(w_t) - \nabla f(w^*) ||_{D_t^{-1}}^2 + 4\eta^2 ||\nabla r(w_t) \\
    & \quad  - \nabla r(w_*) ||_{D_t}^2 + 4\eta^2||\nabla r(w_*)||_{D_t}^2 \\
    &\leq 4\eta^2\lambda^2 ||w_t - w_*||_{D_t}^2 + \frac{4\eta^2L_f}{\alpha} \left(f(w_t)-f(w_*)\right) \\
    & \quad + 4\eta^2\lambda^2\Omega_0^2||D_t||_2 \\
    & \leq 4\eta^2\lambda^2 ||w_t - w_*||_{D_t}^2 + \frac{4\eta^2L_f}{\alpha} \left(f(w_t)-f(w_*)\right) \\
    & \quad + 4\eta^2 \frac{\lambda^2\Omega_0^2}{\alpha} ||D_t||_2^2.
\end{split}
\end{equation*}
Наконец, используя дополнительные обозначения для $R_{t+1}^2 = ||w_{t+1} - w_*||_{D_{t}}^2$ и леммы \ref{lemma:Dt} об изменении $D_t$, мы получаем:
\begin{equation*}
\begin{split}
    R_{t+1}^2 &\leq \left(1 - \eta\mu_f - \eta\lambda+4\eta^2\lambda^2 \right)\left(1+\frac{(1-\beta)\Gamma^2}{2\alpha}\right)R_t^2 + \left( \frac{4\eta^2\lambda^2\Omega_0^2}{\alpha} + \frac{\lambda\eta\Omega_0^2}{\alpha}\right) \sigma_t^2 \\
    & + \left(\frac{4\eta^2L_f}{\alpha} - 2\eta\right) \left(f(w_t) - f(w_*) \right).
\end{split}
\end{equation*}
\begin{equation*}
    \sigma_{t+1}^2 \leq \beta \sigma_{t}^2 + 2\frac{1-\beta}{\beta^2}L_f (f(w_t) - f(w_*)).
\end{equation*}
\begin{equation*}
\begin{split}
    R_{t+1}^2 + M\sigma_{t+1}^2 &\leq  \left(1 - \eta\mu_f - \eta\lambda + 4\eta^2\lambda^2 \right)\left(1+\frac{(1-\beta)\Gamma^2}{2\alpha}\right)R_t^2 \\
    & \quad + \left( \frac{4\eta^2\lambda^2\Omega_0^2}{\alpha} + \frac{\lambda\eta\Omega_0^2}{\alpha} + M\beta\right) \sigma_t^2 \\
    & \quad + \left( \frac{4\eta^2L_f}{\alpha} + 2M\frac{1-\beta}{\beta^2}L_f - 2\eta\right) \left(f(w_t) - f(w_*) \right).
    \end{split}
\end{equation*}
Напишем ограничения на шаг алгоритма, то есть на  $\eta$: 
$$\beta \geq 1 - \frac{\eta(\mu_f+\lambda)\alpha}{2\Gamma^2},$$ 
$$\eta \leq \frac{\mu_f + \lambda}{8 \lambda^2}$$
\begin{equation*}
\begin{split}
    \left(1 - \eta\mu_f - \eta\lambda + 4\eta^2\lambda^2 \right)\left(1+\frac{(1-\beta)\Gamma^2}{2\alpha}\right) &\leq \left(1 - \eta \frac{\mu_f+\lambda}{2}\right) \left(1 + (1-\beta) \frac{\Gamma^2}{2\alpha} \right) \\
     &\leq \left(1 - \eta \frac{\mu_f+\lambda}{2}\right) \left(1 + \eta\frac{\mu_f + \lambda}{4} \right) \\
     & = 1 + \eta \frac{\mu_f + \lambda}{4} - \eta \frac{\mu_f + \lambda}{2} - \eta^2 \frac{(\mu_f+\lambda)^2}{8} \\
     & = 1 - \eta \frac{\mu_f + \lambda}{4} - \eta^2 \frac{(\mu_f + \lambda)^2}{8} \\
     & < 1 - \eta \frac{\mu_f + \lambda}{4}.
    \end{split}
\end{equation*}
Запишем ограничения на второй множитель выражения, причем ещё ограничение на шаг обучения  $\eta < \frac{1}{4\lambda}$:
\begin{equation*}
    \left( \frac{4\eta^2\lambda^2\Omega_0^2}{\alpha} + \frac{\lambda\eta\Omega_0^2}{\alpha} + M\beta\right) \leq \frac{2\lambda\eta\Omega_0^2}{\alpha} + M\beta = \left(\frac{1+\beta}{2} \right)M.
\end{equation*}
\begin{equation*}
    M = \frac{4\eta \lambda \Omega_0^2}{\alpha(1-\beta)}.
\end{equation*}
Запишем ограничения на третий множитель:
\begin{equation*}
    2\eta^2 \frac{L_f}{\alpha} - \eta + \frac{1-\beta}{\beta^2}L_fM = 2\eta^2 \frac{L_f}{\alpha} - \eta + \frac{1-\beta}{\beta^2}L_f \frac{4\eta \lambda \Omega_0^2}{\alpha(1-\beta)} \leq 0.
\end{equation*}
Поделим обе части на $\eta$:
\begin{equation*}
    2\eta \frac{L_f}{\alpha} - 1 + \frac{1-\beta}{\beta^2}L_f \frac{4 \lambda \Omega_0^2}{\alpha(1-\beta)} \leq 0.
\end{equation*}
С ограничениями на $\lambda:$ 
\begin{equation*}
\lambda \leq \frac{\alpha \beta^2}{8L_f\Omega_0^2},
\end{equation*}
получили, что
\begin{equation*}
    \eta < \frac{\alpha}{4 L_f} \leq \frac{\alpha}{2L_f} \left(1 - \frac{4L_f\lambda\Omega_0^2}{\alpha\beta^2} \right).
\end{equation*}
Наконец, мы можем запустить рекурсию:
\begin{equation*}
\begin{split}
    R_{T+1}^2 + M \sigma_{T+1}^2 & \leq \left(1-\eta\frac{\mu_f+\lambda}{4} \right) R_T^2 + \left(\frac{1+\beta}{2}\right) \cdot M \sigma_T^2 \\
    & \leq \exp{\left(- \min \left\{\eta\frac{\mu_f + \lambda}{4}; -\log\left(\frac{1+\beta}{2} \right) \right\} \right)} \left( R_T^2 + M \sigma_T^2 \right) \\
    & \leq \exp{\left(- \min \left\{\eta\frac{\mu_f + \lambda}{4};\eta\frac{(\mu_f + \lambda)\alpha}{2\Gamma^2}  \right\} \right)} \left( R_T^2 + M \sigma_T^2 \right) \\
    &\leq \exp{\left(-T \eta\frac{\mu_f + \lambda}{4} \cdot \min \left\{1; \frac{2\alpha}{\Gamma^2} \right\} \right)}\left( R_0^2 + M\sigma_0^2 \right).
    \end{split}
\end{equation*}
У нас есть список ограничений на гиперпараметры алгоритма:
\begin{enumerate}
    \item $\lambda < \frac{\alpha \beta^2}{8L_f \Omega_0^2}$.
    \item $\eta \leq \frac{8\mu_f L_f^2 \Omega_0^4}{\alpha^2 \beta^4} + \frac{L_f \Omega_0^2}{\alpha \beta^2} < \frac{\mu_f + \lambda}{8\lambda^2}$.
    \item $\eta < \frac{2L_f \Omega_0^2}{\alpha \beta^2} \leq \frac{1}{4\lambda}$.
    \item $\beta \geq 1 - \frac{\eta(\mu_f + \lambda)\alpha}{2 \Gamma^2}$.
    \item $\eta < \frac{\alpha}{4L_f}$.
\end{enumerate}
\begin{equation*}
    \eta_{min} = \min \left\{ \frac{2L_f \Omega_0^2}{\alpha \beta^2}; \frac{\alpha}{4L_f}; \frac{8\mu_f L_f^2 \Omega_0^4}{\alpha^2 \beta^4} + \frac{L_f \Omega_0^2}{\alpha \beta^2} \right\}.
\end{equation*}
Получили оценку количества шагов, необходимых для достижения заданной точности
\begin{equation*}
T = \mathcal{O} \left(\log\left(\frac{R_0^2 + \frac{8\lambda \Omega_0^2 \Gamma^2}{\alpha^2(\mu_f+\lambda)} \sigma_0^2}{\varepsilon} \right) \frac{4}{\eta_{min} (\mu_f + \lambda)\cdot \min \left\{1; \frac{2\alpha}{\Gamma^2} \right\} } \right)    
\end{equation*}
\end{proof}
